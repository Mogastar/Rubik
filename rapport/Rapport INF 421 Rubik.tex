\documentclass{article}
\usepackage[utf8]{inputenc} 
\title{Réseudre de manière optimale le Rubik’s cube} 
\author{Yuxiang LI, Kevin CHEN}
\begin{document} 
\maketitle 

\paragraph{Introduction : }
\section{Configuration}
\paragraph{}
\subsection{Classes et constructeurs} 
\paragraph{}
\subsubsection{Cube} 
\paragraph{}
\subsubsection{Coin / Edge} 
\paragraph{}
\subsubsection{Action}
\paragraph{} 
\subsubsection{Chemin}
\paragraph{}
\subsubsection{Test}
\paragraph{}
\subsection{Implémentation de la rotation} 
\paragraph{Enumération} est la méthode la plus simple à réaliser. 

\section{Recherche exhausitive}
\subsection{Parcours en largeur}
\subsubsection{Nombre de dispostions possibles}
\paragraph{}
\subsubsection{Limite de la recherche BFS}
\paragraph{}
\subsection{Amélioration par l’algorithme A*}
\subsubsection{Condition d’utilisation de la fonction estimatrice}
\paragraph{}
\subsubsection{Implémentation de l’estimateur de distance}
\paragraph{}
\subparagraph{Distance de Manhattan}
\subparagraph{Distance réelle}
\subsubsection{Comparaison avec la recherche BFS}
\paragraph{}

\section{Méthode IDA* et pattern database}
\subsection{Implémentation de la méthode}
\paragraph{}
\subsection{Résultat}
\paragraph{}

\section{Conclusion}
\paragraph{}

\end{document} 